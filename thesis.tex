
%Remove these comments at end if they are fulfilled

% Before you begin, make sure that:
% - research question is clear
% - main points(contribution)+goal are clear (to be repeated throughout+title)
% - clear validation, evaluation, method of exactly the research question with main points as outcome!!!
% - Brainstorming was done
% - Roter Faden was done (Strukturierung und Elimination of Abzweigung)



\iffalse
NOTES:
- Why TOML?
	* Used by many projects
		-> eg cargo, pip -> where exactly
		-> possiblities for usage of elektra
	* No TOML plugin existing (only via augeas plugin)
- What was done?
	* TOML storage plugin for elektra
	* For now limited to reading toml files
	* uses c, flex, bison
- Case Study
	* lcdproc

	
\fi

\documentclass[12pt]{report}
\usepackage[utf8]{inpute nc}
\usepackage[acronym,nomain]{glossaries}
\usepackage{cite}
\usepackage{hyperref}

\title{Viability of TOML parsing for configuration}
\author{Jakob Fischer}


\makeglossaries

\newacronym{LL}{LL}{Left-to-Right, Leftmost derivation}
\newacronym{TOML}{TOML}{Tom's Obvious, Minimal Language}

\newcommand{\onlinesrc}[1]{Accessed #1}

\begin{document}

\maketitle

\begin{abstract}
TODO: Abstract
\end{abstract}

\chapter{Introduction}

\section{Research Question}

\section{\acrshort{TOML} (\acrlong{TOML})}
\acrshort{TOML} is a configuration file format developed since February, 2013 \cite{tomlcontrib}.
It claims to be a format, that is easily to read and parse \cite{tomlreadme} and finds use in many different projects \cite{tomlwiki}.
For instance, it is used with Rusts project manager \textbf{cargo} \cite{cargogit}, as well as Pythons package installer \textbf{pip} \cite{piprefguide}.

\section{Elektra}
Elektra is a configuration framework for accessing configuration settings in a global key database \cite{elektramain}.
An application using Elektra can read and write key/value pairs via simple calls to the Elektra library, making the implementation of an own configuration system obsolete.
Furthermore, elektrified applications can access configuration settings of other elektrified applications, to provide better application interoperability.

Although Elektra is mainly written in C, there are bindings for other languages like java, python or ruby \cite{elektrabindings}.

Elektra can also read from and write to different configuration file formats, like JSON, XML or ini \cite{elektrastorage}.
If the need arises, applications can easily switch to a different file format, because configuration access is done with the Elektra API.
Developers or Administrators no longer need to commit to one file format for configuration.

Support for different languages and file formats is implemented by the use of plugins, which can be enabled if needed.
With this modularity, Elektra can provide a wide range of functionality, while simultaneously avoiding being a bloated library.
Developers can compile Elektra with the exact set of needed functionality. If they don't need bindings for java, they can just disable this feature.

\section{lcdproc}



\chapter*{Methodology}

\chapter*{Implementation}

\chapter*{Evaluation}

\chapter*{Related Work}

\chapter*{Conclusion}

\chapter*{Glossary}

\printglossary[type=\acronymtype]

\bibliography{references}{}
\bibliographystyle{plain}

\end{document}
